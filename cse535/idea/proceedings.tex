\documentclass{sigchi}

% Use this section to set the ACM copyright statement (e.g. for
% preprints).  Consult the conference website for the camera-ready
% copyright statement.

% Copyright
\CopyrightYear{2016}
%\setcopyright{acmcopyright}
\setcopyright{acmlicensed}
%\setcopyright{rightsretained}
%\setcopyright{usgov}
%\setcopyright{usgovmixed}
%\setcopyright{cagov}
%\setcopyright{cagovmixed}
% DOI
\doi{http://dx.doi.org/10.475/123_4}
% ISBN
\isbn{123-4567-24-567/08/06}
%Conference
\conferenceinfo{CHI'16,}{May 07--12, 2016, San Jose, CA, USA}
%Price
\acmPrice{\$15.00}

% Use this command to override the default ACM copyright statement
% (e.g. for preprints).  Consult the conference website for the
% camera-ready copyright statement.

%% HOW TO OVERRIDE THE DEFAULT COPYRIGHT STRIP --
%% Please note you need to make sure the copy for your specific
%% license is used here!
% \toappear{
% Permission to make digital or hard copies of all or part of this work
% for personal or classroom use is granted without fee provided that
% copies are not made or distributed for profit or commercial advantage
% and that copies bear this notice and the full citation on the first
% page. Copyrights for components of this work owned by others than ACM
% must be honored. Abstracting with credit is permitted. To copy
% otherwise, or republish, to post on servers or to redistribute to
% lists, requires prior specific permission and/or a fee. Request
% permissions from \href{mailto:Permissions@acm.org}{Permissions@acm.org}. \\
% \emph{CHI '16},  May 07--12, 2016, San Jose, CA, USA \\
% ACM xxx-x-xxxx-xxxx-x/xx/xx\ldots \$15.00 \\
% DOI: \url{http://dx.doi.org/xx.xxxx/xxxxxxx.xxxxxxx}
% }

% Arabic page numbers for submission.  Remove this line to eliminate
% page numbers for the camera ready copy
% \pagenumbering{arabic}

% Load basic packages
\usepackage{balance}       % to better equalize the last page
\usepackage{graphics}      % for EPS, load graphicx instead 
\usepackage[T1]{fontenc}   % for umlauts and other diaeresis
\usepackage{txfonts}
\usepackage{mathptmx}
\usepackage[pdflang={en-US},pdftex]{hyperref}
\usepackage{color}
\usepackage{booktabs}
\usepackage{textcomp}

% Some optional stuff you might like/need.
\usepackage{microtype}        % Improved Tracking and Kerning
% \usepackage[all]{hypcap}    % Fixes bug in hyperref caption linking
\usepackage{ccicons}          % Cite your images correctly!
% \usepackage[utf8]{inputenc} % for a UTF8 editor only

% If you want to use todo notes, marginpars etc. during creation of
% your draft document, you have to enable the "chi_draft" option for
% the document class. To do this, change the very first line to:
% "\documentclass[chi_draft]{sigchi}". You can then place todo notes
% by using the "\todo{...}"  command. Make sure to disable the draft
% option again before submitting your final document.
\usepackage{todonotes}

% Paper metadata (use plain text, for PDF inclusion and later
% re-using, if desired).  Use \emtpyauthor when submitting for review
% so you remain anonymous.
\def\plaintitle{CSE 535 IDEA PAPER}
\def\plainauthor{First Author, Second Author, Third Author,
  Fourth Author, Fifth Author, Sixth Author}
\def\emptyauthor{}
\def\plainkeywords{Authors' choice; of terms; separated; by
  semicolons; include commas, within terms only; required.}
\def\plaingeneralterms{Documentation, Standardization}

% llt: Define a global style for URLs, rather that the default one
\makeatletter
\def\url@leostyle{%
  \@ifundefined{selectfont}{
    \def\UrlFont{\sf}
  }{
    \def\UrlFont{\small\bf\ttfamily}
  }}
\makeatother
\urlstyle{leo}

% To make various LaTeX processors do the right thing with page size.
\def\pprw{8.5in}
\def\pprh{11in}
\special{papersize=\pprw,\pprh}
\setlength{\paperwidth}{\pprw}
\setlength{\paperheight}{\pprh}
\setlength{\pdfpagewidth}{\pprw}
\setlength{\pdfpageheight}{\pprh}

% Make sure hyperref comes last of your loaded packages, to give it a
% fighting chance of not being over-written, since its job is to
% redefine many LaTeX commands.
\definecolor{linkColor}{RGB}{6,125,233}
\hypersetup{%
  pdftitle={\plaintitle},
% Use \plainauthor for final version.
%  pdfauthor={\plainauthor},
  pdfauthor={\emptyauthor},
  pdfkeywords={\plainkeywords},
  pdfdisplaydoctitle=true, % For Accessibility
  bookmarksnumbered,
  pdfstartview={FitH},
  colorlinks,
  citecolor=black,
  filecolor=black,
  linkcolor=black,
  urlcolor=linkColor,
  breaklinks=true,
  hypertexnames=false
}

% create a shortcut to typeset table headings
% \newcommand\tabhead[1]{\small\textbf{#1}}

% End of preamble. Here it comes the document.
\begin{document}

\title{\plaintitle}

\numberofauthors{5}
\author{%
  \alignauthor{Matthew Howard\\
    \affaddr{Odenton MD, USA}\\
    \email{mshowar4@asu.edu}}\\
  \alignauthor{Leave Authors Anonymous\\
    \affaddr{for Submission}\\
    \affaddr{City, Country}\\
    \email{e-mail address}}\\
  \alignauthor{Leave Authors Anonymous\\
    \affaddr{for Submission}\\
    \affaddr{City, Country}\\
    \email{e-mail address}}\\
  \alignauthor{Leave Authors Anonymous\\
    \affaddr{for Submission}\\
    \affaddr{City, Country}\\
    \email{e-mail address}}\\
  \alignauthor{Leave Authors Anonymous\\
    \affaddr{for Submission}\\
    \affaddr{City, Country}\\
    \email{e-mail address}}\\
}

\maketitle

\begin{abstract}
	This is the abstract
\end{abstract}

\category{H.5.m.}{Information Interfaces and Presentation
  (e.g. HCI)}{Miscellaneous} \category{See
  \url{http://acm.org/about/class/1998/} for the full list of ACM
  classifiers. This section is required.}{}{}

\keywords{\plainkeywords}
\section{Introduction}

Precision agriculture and TinyML have transformed commercial farming, but their extension to home and backyard gardens remains limited. This paper demonstrates that by leveraging low-cost IoT sensors and edge-ML models, precision agriculture can now be \emph{approachable, practical, and sustainable} for home gardeners within constraints of time, budget, and minimal monitoring. The framework supports automation (real-time irrigation, nutrient management), accessibility (easy deployment for non-experts), and sustainability (waste reduction, resource efficiency), offering a scalable blueprint for home-scale smart agriculture.

\section{IoT Sensing and Monitoring}

Traditional soil assessment is reactive and slow. Continuous, real-time monitoring via IoT sensors—moisture, pH, nutrient (NIR), EC for fertilizer salts, and plant stress proxies like temperature and humidity—enables timely intervention and optimal resource use. Low-cost hardware (e.g., <\$20/probe, <\$10 Zigbee radios) makes distributed sensor networks practical for home users. These compact devices provide immediate feedback, fostering proactive soil care and minimizing input waste.

\section{Networking: Zigbee for Home Gardens}

Zigbee’s low-power, local wireless communication is well-suited to household plots, offering mesh networking, low module cost, and no subscription fees. Sensors and actuators (pumps, fans) communicate efficiently, and the flexible mesh topology allows easy expansion to new zones without complex rewiring or reliance on cloud connectivity.

\section{From Data to Action: ML and AI Integration}

Continuous sensor data streams are translated by lightweight ML models on edge gateways into actionable insights—detecting trends in soil or plant health, predicting irrigation and nutrient needs, classifying soil status, and identifying anomalies. The AI decision layer coordinates these outputs, triggering pumps, fans, or fertilizer recommendations at the optimal time, establishing a \emph{closed-loop system} for sustainable, hands-off gardening.

\section{Home-Scale Applications and Outcomes}

This synergistic approach enables:
\begin{itemize}
	\item Smart irrigation based on real-time drying prediction rather than fixed schedules.
	\item Automated compost or fertilizer guidance, reducing nutrient waste and runoff.
	\item Simple climate regulation in hobby greenhouses using affordable sensor-triggered ventilation.
	\item User-friendly restoration and rotation recommendations with clear, actionable feedback.
\end{itemize}
Recent studies confirm that such systems, previously exclusive to industrial farms, now fit seamlessly in home environments thanks to edge computing, affordable hardware, and robust local networking.

\section{Conclusion}

Affordable IoT and AI-driven systems make precision agriculture a reality for home gardeners. The integration of real-time sensors, edge ML, and AI orchestrates proactive soil and plant management, reducing costs and environmental impact in small, accessible garden settings.


= Relevance

The growing demand for sustainable and efficient farming introduced by population increases, global climate change, and supply chain disruptions makes local sustainable farming more pressing than ever. However, the issues of power consumption and latency exist for virtually all IoT systems.  

The frequent problems faced in many Wireless Sensor and Actuator Networks (WSANs) revolve around power and computational restraints. Traditionally, with the onset of cloud computing, the generally accepted model was to offload computation to the cloud. However, in the case of home farming, concerns of latency and power consumption become more pronounced. Offloading to a nearby fog device or using a Zigbee network could account for the computational capabilities that sensors and actuators are unable to handle, while still being close enough in proximity to reduce latency concerns.

= Advantages in Power and Latency

Similar analysis was done by Panday [1] in a study pertaining to soil analysis. The findings point to reduced latency and power consumption when implementing a fog device. We have expanded this to compare these findings to the latency and power demands of cloud and TinyML systems.

#figure(caption: [_Comparison of TinyML, Edge (Gateway), and Cloud._], kind: table)[
#text(size: 8pt)[
  #table(
    columns: (1fr, 1fr, 1fr, 1fr),
    align: left,
    stroke: 0.5pt + luma(220),
    inset: 3pt,
    table.header(
      [*Criterion*], [*TinyML Sensor*], [*EDGE (Gateway)*], [*Cloud*]
    ),
    [Latency / timeliness], [Good (local, limited)], [Excellent: < 100 ms, multi-sensor], [Variable (200 msÐseconds)],
    [Power use], [Higher (ML + radio)], [Best: sensors sleep, gateway mains], [Sensors must stay chatty, added power of cloud consumption],
    [Model complexity], [Very limited (RAM/flash)], [Moderate (RF, SVM, ARIMA feasible)], [Any size],
    [Updates], [Hard (reflash nodes)], [Easy (one box to update)], [Easy, but depends on net],
    [Connectivity], [None], [None (local-first)], [High dependency],
    [Fit for 1Ð10 acres], [Overkill], [Best choice], [Overkill]
  )
]
]

These findings, sourced partly from Panday [1], demonstrate how fog-based models can outperform TinyML and cloud-based models across multiple domains. Particularly, edge-based models are shown to perform with under 100 ms of latency, compared to expected latency of over 200 ms for cloud-based devices.

= Power Comparisons

Additionally, further benefits in power consumption can be seen with an edge-based ML paradigm. Bonomi [2] analyzes the benefits of pairing an ML model with a fog device in a WSAN-based system. In this case, ML models can learn when actuators and sensors need to perform a task, and when they are allowed to rest.  

There was a noticeable block in advancement of systems that introduced actuators, since anything that goes beyond sensing and tracking introduced power requirements, greatly increasing the consumption of power. They note that the introduction of fog devices can help relieve the power shortages that are presented in more complex architectures.


\balance{}

\section{References Format}
gonline videos~\cite{psy:gangnam} is given here.
% BALANCE COLUMNS
\balance{}

% REFERENCES FORMAT
% References must be the same font size as other body text.
\bibliographystyle{SIGCHI-Reference-Format}
\bibliography{sample}

\end{document}

%%% Local Variables:
%%% mode: latex
%%% TeX-master: t
%%% End:
