\documentclass{sigchi}

% Use this section to set the ACM copyright statement (e.g. for
% preprints).  Consult the conference website for the camera-ready
% copyright statement.

% Copyright
\CopyrightYear{2016}
%\setcopyright{acmcopyright}
\setcopyright{acmlicensed}
%\setcopyright{rightsretained}
%\setcopyright{usgov}
%\setcopyright{usgovmixed}
%\setcopyright{cagov}
%\setcopyright{cagovmixed}
% DOI
\doi{http://dx.doi.org/10.475/123_4}
% ISBN
% \isbn{123-4567-24-567/08/06}
%Conference
% \conferenceinfo{CHI'16,}{May 07--12, 2016, San Jose, CA, USA}
%Price
% \acmPrice{\$15.00}

% Use this command to override the default ACM copyright statement
% (e.g. for preprints).  Consult the conference website for the
% camera-ready copyright statement.

%% HOW TO OVERRIDE THE DEFAULT COPYRIGHT STRIP --
%% Please note you need to make sure the copy for your specific
%% license is used here!
% \toappear{
% Permission to make digital or hard copies of all or part of this work
% for personal or classroom use is granted without fee provided that
% copies are not made or distributed for profit or commercial advantage
% and that copies bear this notice and the full citation on the first
% page. Copyrights for components of this work owned by others than ACM
% must be honored. Abstracting with credit is permitted. To copy
% otherwise, or republish, to post on servers or to redistribute to
% lists, requires prior specific permission and/or a fee. Request
% permissions from \href{mailto:Permissions@acm.org}{Permissions@acm.org}. \\
% \emph{CHI '16},  May 07--12, 2016, San Jose, CA, USA \\
% ACM xxx-x-xxxx-xxxx-x/xx/xx\ldots \$15.00 \\
% DOI: \url{http://dx.doi.org/xx.xxxx/xxxxxxx.xxxxxxx}
% }

% Arabic page numbers for submission.  Remove this line to eliminate
% page numbers for the camera ready copy
% \pagenumbering{arabic}

% Load basic packages
\usepackage{balance}       % to better equalize the last page
\usepackage{graphics}      % for EPS, load graphicx instead 
\usepackage[T1]{fontenc}   % for umlauts and other diaeresis
\usepackage{txfonts}
\usepackage{mathptmx}
\usepackage[pdflang={en-US},pdftex]{hyperref}
\usepackage{color}
\usepackage{booktabs}
\usepackage{textcomp}
\usepackage{dblfloatfix}

% Some optional stuff you might like/need.
\usepackage{microtype}        % Improved Tracking and Kerning
% \usepackage[all]{hypcap}    % Fixes bug in hyperref caption linking
\usepackage{ccicons}          % Cite your images correctly!
% \usepackage[utf8]{inputenc} % for a UTF8 editor only

% If you want to use todo notes, marginpars etc. during creation of
% your draft document, you have to enable the "chi_draft" option for
% the document class. To do this, change the very first line to:
% "\documentclass[chi_draft]{sigchi}". You can then place todo notes
% by using the "\todo{...}"  command. Make sure to disable the draft
% option again before submitting your final document.
\usepackage{todonotes}

% Paper metadata (use plain text, for PDF inclusion and later
% re-using, if desired).  Use \emtpyauthor when submitting for review
% so you remain anonymous.
\def\plaintitle{CSE 535 IDEA PAPER}
\def\plainauthor{First Author, Second Author, Third Author,
  Fourth Author, Fifth Author, Sixth Author}
\def\emptyauthor{}
\def\plainkeywords{Authors' choice; of terms; separated; by
  semicolons; include commas, within terms only; required.}
\def\plaingeneralterms{Documentation, Standardization}

% llt: Define a global style for URLs, rather that the default one
\makeatletter
\def\url@leostyle{%
  \@ifundefined{selectfont}{
    \def\UrlFont{\sf}
  }{
    \def\UrlFont{\small\bf\ttfamily}
  }}
\makeatother
\urlstyle{leo}

% To make various LaTeX processors do the right thing with page size.
\def\pprw{8.5in}
\def\pprh{11in}
\special{papersize=\pprw,\pprh}
\setlength{\paperwidth}{\pprw}
\setlength{\paperheight}{\pprh}
\setlength{\pdfpagewidth}{\pprw}
\setlength{\pdfpageheight}{\pprh}

% Make sure hyperref comes last of your loaded packages, to give it a
% fighting chance of not being over-written, since its job is to
% redefine many LaTeX commands.
\definecolor{linkColor}{RGB}{6,125,233}
\hypersetup{%
  pdftitle={\plaintitle},
% Use \plainauthor for final version.
%  pdfauthor={\plainauthor},
  pdfauthor={\emptyauthor},
  pdfkeywords={\plainkeywords},
  pdfdisplaydoctitle=true, % For Accessibility
  bookmarksnumbered,
  pdfstartview={FitH},
  colorlinks,
  citecolor=black,
  filecolor=black,
  linkcolor=black,
  urlcolor=linkColor,
  breaklinks=true,
  hypertexnames=false
}

% create a shortcut to typeset table headings
% \newcommand\tabhead[1]{\small\textbf{#1}}

% End of preamble. Here it comes the document.
\begin{document}

\title{\plaintitle}

\numberofauthors{6}
\author{%
  \alignauthor{Matthew Howard\\
    \affaddr{Odenton, MD}\\
    \email{mshowar4@asu.edu}}\\
  \alignauthor{Eshwar Pitchiah\\
    \affaddr{Queen Creek, AZ}\\
    \email{epitchia@asu.edu}}\\
  \alignauthor{Carson Darling\\
    \affaddr{Scottsdale, AZ}\\
    \email{cdarlin1@asu.edu}}\\
  \alignauthor{Priyananda Vangala\\
    \affaddr{St. Louis, MO}\\
    \email{pvangal1@asu.edu}}\\
  \alignauthor{Trevor Webster\\
    \affaddr{williston, VT}\\
    \email{twebst10@asu.edu}}\\
  \alignauthor{Calvin Peng}\\
    \affaddr{San Francisco, CA}\\
    \email{ypeng97@asu.edu}
}

\maketitle

\begin{abstract}
This paper has lined out a five year plan that provides an efficient and scalable home smart agriculture system using mobile computing technology that will identify and resolve issues surrounding urban food security in a sustainable manner. Using IoT sensors, mesh networks, and machine learning, we will show that it is possible to monitor crops and soil in real time with minimal latency, cost, and power consumption. Our solution is superior to current commercial farming solutions because we approach the issue via home scaled deployments rather than in an industrial manner by extending TinyML and IoT capabilities to farmers with limited experience in new technologies, which allow for a larger base of users for the product rather than limiting to bigger players in the agribusiness space. Over the course of our five year plan, we will progressively establish real time CPS loops, integrate external data, harden the security protocols in our system, and finalize interoperability standards all while conducting pilots. This process will overall demonstrate how the use of mobile technology in this unique manner can transform home gardening forever and take us closer to a future where agriculture will be a resilient, automated, and environmentally conscious practice.
\end{abstract}

\section{Introduction}

As climate change intensifies and the world population rapidly grows, we see a widespread drop in local and global-scale food production. Food, and more importantly healthy food, is a necessity for any thriving community, and agriculture remains a central component for all sustainable development of a future society. While home gardening currently holds significant promise a solution to the growing shortage of food without impacting the environment negatively, we lack the infrastructure to scale up the application in a meaningful way. This gap can, and should, be solved through the use of automation and intelligence so that we can optimize this resource use at a larger scale. Precision agriculture technology remains largly inaccessible to local amateur and small scale gardeners due to the high operating costs and technical complexity it brings. We can provide a viable solution to this gap by proposing a five year development plan that combines affordable IoT sensors, Zigbee mesh networking, and edge ML to create a mobile computing–driven smart farming framework tailored to assist and augment home gardens.  In this plan, we will outline the steps that can be taken to implement our solution at a large scale while emphasizing the significance, relevance, novelty, soundness, and feasibility of our plan to make intelligent home farming a sensible and secure practice.


\section{Significance}

Home garden smart farming is important because it empowers individuals and households to use technology for producing food in a sustainable and efficient way. By integrating sensors, automated irrigation, and small-scale data analytics, home gardeners can monitor soil moisture, light levels, and crop health to optimize input use and reduce waste\cite{wolfert2017big}. These systems not only conserve water and energy but also improve yields in limited spaces, which is especially valuable in urban areas. In addition, applying big data and smart tools at the home garden level increases resilience by enabling people to adapt to changing weather conditions and reduce dependence on industrial supply chains\cite{elbilali2018smart}.

\subsection{Social Impact}

Beyond efficiency, home garden smart farming is significant for its social and environmental impacts. It allows households to grow healthier, pesticide-free food while contributing to local food security and sustainability goals \cite{elbilali2018smart}. By scaling down principles of smart farming, individuals can participate in the broader shift toward data-driven agriculture and sustainable development. Home garden systems also serve as educational platforms, raising awareness about resource management, food safety, and the role of technology in creating resilient food systems\cite{wolfert2017big}. In this way, home garden smart farming links personal well-being with global sustainability by combining innovation, self-sufficiency, and environmental care.

\section{Novelty}

Precision agriculture and TinyML have transformed commercial farming, but their extension to home and backyard gardens remains limited. This paper demonstrates that by leveraging low-cost IoT sensors and edge-ML models, precision agriculture can now be \emph{approachable, practical, and sustainable} for home gardeners within constraints of time, budget, and minimal monitoring. The framework supports automation (real-time irrigation, nutrient management), accessibility (easy deployment for non-experts), and sustainability (waste reduction, resource efficiency), offering a scalable blueprint for home-scale smart agriculture.

Traditional soil assessment is reactive and slow. Continuous, real-time monitoring via IoT sensors—moisture, pH, nutrient (NIR), EC for fertilizer salts, and plant stress proxies like temperature and humidity—enables timely intervention and optimal resource use. Low-cost hardware (e.g., <\$20/probe, <\$10 Zigbee radios) makes distributed sensor networks practical for home users. These compact devices provide immediate feedback, fostering proactive soil care and minimizing input waste.

\subsection{Networking: Zigbee for Home Gardens}

Zigbee’s low-power, local wireless communication is well-suited to household plots, offering mesh networking, low module cost, and no subscription fees. Sensors and actuators (pumps, fans) communicate efficiently, and the flexible mesh topology allows easy expansion to new zones without complex rewiring or reliance on cloud connectivity.

\subsection{From Data to Action: ML and AI Integration}

Continuous sensor data streams are translated by lightweight ML models on edge gateways into actionable insights—detecting trends in soil or plant health, predicting irrigation and nutrient needs, classifying soil status, and identifying anomalies. The AI decision layer coordinates these outputs, triggering pumps, fans, or fertilizer recommendations at the optimal time, establishing a \emph{closed-loop system} for sustainable, hands-off gardening.

\subsection{Home-Scale Applications and Outcomes}

This synergistic approach enables:
\begin{itemize}
	\item Smart irrigation based on real-time drying prediction rather than fixed schedules.
	\item Automated compost or fertilizer guidance, reducing nutrient waste and runoff.
	\item Simple climate regulation in hobby greenhouses using affordable sensor-triggered ventilation.
	\item User-friendly restoration and rotation recommendations with clear, actionable feedback.
\end{itemize}
Recent studies confirm that such systems, previously exclusive to industrial farms, now fit seamlessly in home environments thanks to edge computing, affordable hardware, and robust local networkinge

\section{Relevance}
\begin{table*}[btp]
	\centering
	\begin{tabular}{l p{4cm} p{4cm} p{4cm}}
		& \multicolumn{3}{c}{\textit{Comparison}} \\
		\cmidrule(r){2-4}
		{\small\textit{Criterion}}
		& {\small\textit{TinyML Sensor}}
		& {\small\textit{EDGE (Gateway)}}
		& {\small\textit{Cloud}} \\
		\midrule
		Latency / timeliness & Good (local, limited) & Excellent: < 100 ms, multi-sensor & Variable (200 ms) \\
		\midrule
		Power use & Higher (ML + radio) & Best: sensors sleep, gateway mains & Sensors must stay chatty, added power of cloud consumption \\
		\midrule
		Model complexity & Very limited (RAM/flash) & Moderate (RF, SVM, ARIMA feasible) & Any size \\
		\midrule
		Updates & Hard (reflash nodes) & Easy (one box to update) & Easy, but depends on net \\
		\midrule
		Connectivity & None & None (local-first) & High dependency \\
		\midrule
		Fit for 1-10 acres & Overkill & Best choice & Overkill \\
	\end{tabular}
\end{table*}

The growing demand for sustainable and efficient farming introduced by population increases, global climate change, and supply chain disruptions makes local sustainable farming more pressing than ever. However, the issues of power consumption and latency exist for virtually all IoT systems.

The frequent problems faced in many Wireless Sensor and Actuator Networks (WSANs) revolve around power and computational restraints. Traditionally, with the onset of cloud computing, the generally accepted model was to offload computation to the cloud. However, in the case of home farming, concerns of latency and power consumption become more pronounced. Offloading to a nearby fog device or using a Zigbee network could account for the computational capabilities that sensors and actuators are unable to handle, while still being close enough in proximity to reduce latency concerns.

\subsection{Advantages in Power and Latency}

Similar analysis was done pertaining to soil analysis. The findings point to reduced latency and power consumption when implementing a fog device. We have expanded this to compare these findings to the latency and power demands of cloud and TinyML systems.

These findings demonstrate how fog-based models can outperform TinyML and cloud-based models across multiple domains. Particularly, edge-based models are shown to perform with under 100 ms of latency, compared to expected latency of over 200 ms for cloud-based devices.

\subsection{Power Comparisons}

Additionally, further benefits in power consumption can be seen with an edge-based ML paradigm. Bonomi\cite{bonomi2012fog},analyzes the benefits of pairing an ML model with a fog device in a WSAN-based system. In this case, ML models can learn when actuators and sensors need to perform a task, and when they are allowed to rest.

There was a noticeable block in advancement of systems that introduced actuators, since anything that goes beyond sensing and tracking introduced power requirements, greatly increasing the consumption of power. They note that the introduction of fog devices can help relieve the power shortages that are presented in more complex architectures.

\section{Soundness}

The soundness of an IoT-enabled gardening system depends on whether the underlying architecture can remain secure, reliable, and adaptable in realistic environments. Prior analyses of IoT systems in agriculture highlight that while the technology is promising, its effectiveness is constrained by vulnerabilities in system design and security protocols. Ray emphasizes that agricultural IoT deployments are often exposed to attack vectors such as eavesdropping on sensor data, denial-of-service against gateways, and unauthorized actuator control, all of which can undermine both trust and operational stability \cite{ray2020security}. For a home garden system, such threats could translate into wasted resources (e.g., malicious overwatering) or damaged plants if functions are hijacked.

Chen builds on these concerns by outlining how end-to-end secure communication frameworks can address such risks in smart farming. Their proposed mechanisms (lightweight cryptographic schemes, mutual authentication between devices, and secure session establishment) form a basis for ensuring data integrity and confidentiality even on resource-constrained IoT nodes \cite{chen2021end}. Applied to home-scale gardening, these principles mean that sensor measurements (such as soil moisture or pH) cannot be tampered with in transit, and actuator commands (such as irrigation triggers) remain authenticated and verifiable.

From a practical standpoint, the soundness of the system is reinforced by three factors. First, the choice of local-first communication (via Zigbee mesh networking and edge-based gateways) minimizes reliance on insecure or high-latency cloud services, reducing exposure to man-in-the-middle attacks \cite{ray2020security}. Second, incorporating lightweight but robust encryption methods ensures that the low-cost IoT devices can still enforce confidentiality and authentication without exhausting power budgets \cite{chen2021end}. Finally, the closed-loop nature of the system—where sensing, decision-making, and actuation occur in a tightly integrated cycle—provides opportunities for anomaly detection. If data streams deviate from expected physical ranges, edge-based ML models can flag potential faults, including those caused by malicious interference.

Taken together, the literature suggests that an IoT gardening application can be considered sound when it adopts security-by-design principles. This includes implementing Ray’s recommendations around resilience against common threats, combined with Chen’s emphasis on end-to-end safeguard mechanisms. By embedding lightweight cryptography, continuous monitoring, and local-first architectures, the system achieves robustness not only in functionality but also in trustworthiness, ensuring that resource efficiency gains are not offset by security risks.

\section{Feasibility}

From Wolfert et al.’s review of smart farming in agriculture, we can gain insights about the constraints that a smart home garden must overcome before implementation \cite{wolfert2017big}. According to the review, smart farming needs to reconfigure under changing garden conditions, pull in external data, meet data-governance requirements, and rely on interoperability/standards for ecosystem uptake. Because a smart home garden is fundamentally similar to smart farming but on a smaller scale, those constraints likely apply. Below is a five-year plan to address the considerations identified in the review.

\subsection{Five Year Plan}

\begin{itemize}

\item Year 1 - Real-time CPS loop using edge or fog. The smart garden must react to real-time events and adjust when conditions change. We will build a simple sensing, decision, and actuation loop on a local hub and check that it responds correctly and safely in common situations such as heat spikes, heavy rain, shifting shade, and sensor faults.

\item Year 2 - External data and robustness. Because smart garden operation depends on external data such as weather and benchmarks, it must keep functioning when feeds are missing, late, or noisy. We will integrate these feeds, add input checks, use caching and fallback rules, log and monitor data quality, and keep core control running locally during outages.

\item Year 3 - Governance and security hardening. Deployment requires clear data ownership and usage terms, strong privacy protections, and security safeguards before broader trials. We will define who can see, keep, and use each data type, set up consent and notices, restrict third-party access, add authentication and encryption for stored and in-transit data, and complete a security review.

\item Year 4 - Pilots and ecosystem fit. Smart garden adoption and reliability must be proven in real homes while platform ecosystems keep changing. We will run pilots in several settings such as balcony pots, backyard beds, and small greenhouses to measure reliability and user satisfaction, integrate with common home hubs where it makes sense, and refine setup and maintenance based on feedback.

\item Year 5 - Interoperability and metadata for scale. To make the smart garden system and its data reusable, they must work with other systems and follow clear standards. We will finish an open data schema, spell out the required metadata such as timestamps, location, calibration, units, and which standards are used, and publish simple API and integration guides so others can replicate and extend the system.

\end{itemize}

\section{Conclusion}

In conclusion, this paper and its ideas will bring precision agriculture to homes across the globe via affordable mobile computing technologies. Our proposed system uses real-time edge processing, Zigbee networks, and local-first security to deliver a viable automation to everyday gardeners in any environment. Our implementation plan for this solution highlights a feasable technical product that has long term potential with easy scalability. This framework empowers everyday commonfolk to contribute to the larger vision of sustainable agriculture by building their own ecosystem to house all their nourishment needs, and can effectively linking individual actions with global change. Ultimately, when we nurture the earth in our own backyards, we plant the seeds too grow a more hopeful and resilient future for everyone.

\balance{}

% \balance{}

\bibliographystyle{SIGCHI-Reference-Format}
\bibliography{sample}

\end{document}

%%% Local Variables:
%%% mode: latex
%%% TeX-master: t
%%% End:
